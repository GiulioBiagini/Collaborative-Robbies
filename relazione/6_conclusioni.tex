\chapter{Conclusioni}
Come abbiamo visto nel capitolo precedente, lasciare agire due robot su una
mappa di 10x10 caselle nella quale sono state sparse 10 lattine, permette di
ottenere ottimi risultati, con un punteggio medio di fitness pari a 95 su 200
diverse sessioni. Questo significa, in media, raccogliere tutte e 10 le lattine
nella metà dei casi e 9 nell'altra.\newline
Abbiamo poi visto come la strategia evolutiva migliore sia quella élitaria,
ovvero, permettere ad un certo numero di robot di passare da una generazione
all'altra e come, non mutando questi individui, siano raggiunti valori ancora
migliori. La percentuale di coppie da lasciare invariata da una generazione alla
successiva è stata poi stimata fra il 10\% ed il 60\%, valori confermati su
popolazioni di 100 e 200 individui.\newline
Usare una vista più grande, per quanto riguarda la non collaborazione (ovvero
impedire ai robot di vedersi sulla mappa), non ha portato i robot né a
raccogliere in media più lattine, quindi a totalizzare un punteggio di fitness
maggiore, né a raccogliere lattine più velocemente (almeno per quanto riguarda
il lasciarli agire per un numero di passi pari al 30\% ed al 40\% di quelli
necessari ad esplorare tutta la mappa: 10x10 = 100 caselle).\newline
Infine, lasciare che i robot possano vedersi sulla mappa ha evidenziato, almeno
per quanto riguarda la strategia evolutiva da noi individuata, come questo possa
rivelarsi un fattore che ostacola l'azione dei due.\newline
Occore in ultima analisi sottolineare come, variando anche in piccola parte
alcuni dei parametri che regolano l'evoluzione, questo porti a grossi
cambiamenti nell'apprendimento dei robot. È dunque possibile che possano
esistere combinazioni di valori che portino anche le coppie che usano viste
``collaborative'' ad ottenere risultati migliori. Ad oggi, però, non siamo
ancora stati in grado di trovarli, o meglio, variare anche la percentuale di
mutazione sembra non essere la strada giusta.



\section{Sviluppi Futuri}
Lasciamo come sviluppi futuri la possibilità di ricercare nuove strategie
evolutive che permettano, se esistono, di evolvere i robot che usano viste
collaborative in modo da ottenere valori di fitness maggiori.\newline
Interessante sarebbe anche introdurre nuove strategie di cooperazione. Con
questo lavoro, infatti, ci siamo solamente dedicati a gettare le basi nello
studio di come possano più entità collaborare nell'azione comune su una mappa.
Abbiamo infatti analizzato le performance di un caso base, quello che usa una
vista a croce nella quale la possibilità di vedere gli altri robot non esiste,
in relazione all'uso di una vista nella quale i due Robby possono vedersi. Ciò
non vieta l'implementazione di ulteriori meccanismi di collaborazione, magari
più complessi, che vedano ad esempio lo scambio di informazioni tra le varie
entità, come la posizione assoluta o relativa nella mappa, o di scambiarsi la
vista, e così via\dots\newline
Infine, comunichiamo ad eventuali interessati che il nostro codice lascia la
possibilità di usare anche una vista quadrata collaborativa, da noi non usata
per le varie prove. Potrebbe essere interessante analizzare se l'uso di tale
vista replichi i risultati ottenuti nella non collaborazione, ovvero, dove usare
una vista più grande non da alcun vantaggio.
