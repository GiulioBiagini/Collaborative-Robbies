\chapter{Progettazione}

\section{Algoritmi Genetici}
Gli \textit{Algoritmi Genetici} sono algoritmi spesso usati in tutte quelle
situazioni nelle quali risulta difficile andare a progettare una determinata
strategia che ci permetta di arrivare ad una soluzione.\newline
Questi algoritmi sono difatti usati per far sì che siano loro stessi ad evolvere
il sistema in modo che esso arrivi autonomamente ad una soluzione. Uno dei campi
nei quali sono maggiormente impiegati, infatti, sono i \textit{Sistemi
Complessi}.\newline
Per far sì che \textbf{Robby} possa muoversi e pulire l'ambiente raccogliendo il
maggior numero di lattine con il minor numero di mosse, non conoscendo a priori
la posizione del robot stesso nella mappa, né delle lattine, non rimane che
progettare un algoritmo genetico che permetta ai robot di apprendere una buona
strategia.\newline
Dapprima sono generati \textit{n} individui con un \textit{dna} casuale, ovvero,
ad ogni possibile vista è associata una mossa casuale. Dopodiché si valuta
ciascun robot per un numero \textit{p} di passi su di una mappa nella quale le
posizioni delle lattine sono generate casualmente, così come la posizione del
robot. Si ripete la valutazione su \textit{m} mappe per ogni robot assegnando,
alla fine, un \textit{valore di fitness} medio a ciascuna entità. Questo valore
è stabilito da una \textit{funzione di fitness} che valuta ogni mossa del robot:
se questo muove contro un ostacolo viene punito con un valore negativo, se tenta
di raccogliere una lattina dove essa non è presente viene altresì puntio,
mentre, nel caso in cui una lattina sia raccolta viene premiato. Alla fine della
fase di valutazione, i robot sono ordinati in base al valore di fitness, ed ha
inizio la generazione della nuova popolazione: sono scelti due individui in base
al valore di fitness o in base alla posizione nel ranking globale e viene
attuato il \textit{crossover}. Questa metodologia consiste nella scelta di un
punto casuale del dna dei due genitori e, unendo rispettivamente la prima metà
del dna del primo genitore con la seconda metà del dna del secondo genitore
viene generato il primo figlio ed unendo la seconda metà del dna del primo
genitore con la prima metà del dna del secondo genitore viene originato il
secondo figlio. Infine sono presi tutti i \textit{geni} (vista/azione) di ogni
figlio e con una probabilità \textit{x} sono \textit{mutati}, cioè, alla vista
è modificata in modo casuale l'azione corrispondente. Questo procedimento di
generazione di una nuova generazione a partire dalla vecchia che viene poi a sua
volta valutata, viene iterato \textit{g} volte. È così che, dopo svariate
generazioni, i robot apprendono strategie di pulizia che permettono loro di
arrivare molto vicino all'obiettivo.
