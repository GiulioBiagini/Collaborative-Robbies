\chapter{Progettazione}
In questo capitolo parleremo di come è stato implementato l'algrotimo genetico
che ha permesso l'apprendimento di una strategia di pulizia alle coppie che
saranno impegnate sulla mappa.

\section{Algoritmi Genetici: schema generale}
Gli \textit{Algoritmi Genetici} sono algoritmi spesso usati in tutte quelle
situazioni nelle quali risulta difficile andare a progettare una determinata
strategia che ci permetta di arrivare ad una soluzione. Questi algoritmi sono
difatti usati per far sì che siano loro stessi ad evolvere il sistema in modo
che esso arrivi autonomamente ad una soluzione. Uno dei campi nei quali sono
maggiormente impiegati, infatti, sono i \textit{Sistemi Complessi}.\newline
Per far sì che \textbf{Robby} possa muoversi e pulire l'ambiente raccogliendo il
maggior numero di lattine con il minor numero di mosse, non conoscendo a priori
la posizione del robot stesso nella mappa, né delle lattine, non rimane che
progettare un algoritmo genetico che permetta ai robot di apprendere una buona
strategia.\newline
La struttura utilizzata nel paper di riferimento~\cite{biblio:robby} è la
seguente:
\begin{itemize}
	\item dapprima sono generati \textit{POPULATION\_SIZE} individui con un dna
	casuale, ovvero, ad ogni possibile vista è associata una mossa scelta a caso
	fra quelle possibili;
	\item dopodiché, per \textit{NUM\_GENERATIONS}:
	\begin{itemize}
		\item viene calcolato il \textit{valore di fitness} per ogni individuo
		assegnando un punteggio 10 nel caso in cui un robot raccolga una
		lattina, -1 se il robot tenta di raccogliere una lattina in una cella
		dove questa non sia presente e -5 punti se il robot tenta di muoversi in
		una casella occupata da un ostacolo (muro). Questa operazione viene
		effettuata per \textit{NUM\_ACTIONS\_PER\_SESSIONS} passi e ripetuta per
		\textit{SESSIONS\_NUMBER} diverse sessioni. Ogni sessione prevede la
		generazione di una mappa nella quale le lattine sono posizionate
		casualmente nelle celle (ogni cella ha la medesima probabilità di
		ospitarne una) così come la posizione del robot è scelta casualmente.
		La mappa ha dimensione 10x10 con 10 lattine. Alla fine, ogni robot
		ottiene un valore di fitness mediato sul numero di sessioni. Il numero
		di azioni su ogni mappa è fissato a 200, così come il numero di
		sessioni;
		\item si procede poi ordinando gli individui in base al proprio valore
		di fitness;
		\item sono scelti due individui nella popolazione con una probabilità
		che varia linearmente in base all'ordinamento (l'individuo con il
		maggiore valore di fitness avrà probabilità più alta di essere scelto
		rispetto al secondo e così via\dots) e viene applicato il
		\textit{crossover}. Questo procedimento genererà due nuovi figli ed è
		ripetuto fino alla generazione della nuova popolazione;
		\item una volta ottenuta la nuova generazione, ogni gene (azione
		corrispondente ad una vista) può essere mutato con una probabilità pari
		a \textit{MUTATION\_PROBABILITY}: 0.5\%.
	\end{itemize}
\end{itemize}

\section{Collaborazione}
Siccome il nostro obiettivo è quello di studiare la collaborazione fra più
robot, nella mappa saranno presenti contemporaneamente 2 agenti. Questo farà sì
che la grandezza POPULATION\_SIZE non indichi il numero di robot nellpopolazione ma il numero di coppie. Dunque, il numero di entità sarà pari al
popolazione ma il numero di coppie. Dunque, il numero di entità sarà pari al
doppio.

\section{Funzione di Fitness}
La presenza di due robot nella mappa comporta anche la modifica dei come la
funzione di fitness valuta gli individui. Il valore di fitness non riguarderà un
singolo agente, ma la coppia.\newline
Nel caso in cui i due robot abbiano la possibilità di vedersi, poi, sarà
assegnato un valore negativo di -5 punti anche nel caso le due entità cerchino
di occupare la stessa cella, scontrandosi, così come avviene nel caso in cui un
robot tenti di muoversi contro il muro. Nel caso in cui non si vedono ed
entrambi occupano la stessa cella nella quale è presente una lattina e cercano
di raccoglierla, il secondo robot non prenderà punteggio negativo in quanto la
lattina è già stata raccolta dal primo. Non si avrà però neanche l'incremento
del valore di fitness di 20 punti (10 per l'azione di raccolta del primo e 10
per l'azione di raccolta del secondo). In un caso simile sarà incrementato il
valore di fitness di soli 10 punti in quanto entrambi hanno fatto l'azione
corretta ma solo una lattina è stata raccolta.

\section{Selezione dei Genitori}
La tecnica da noi usata per generare la nuova popolazione non è esattamente
quella descritta nel paper. Difatti, a seconda delle varie simulazioni, saranno
copiati alcuni individui dalla vecchia alla nuova generazione senza passare
tramite il crossover.

\section{Crossover}
La tecnica di crossover da noi usata è quella descritta nel paper, ovvero: è
scelto un punto di taglio casuale nel dna dei due genitori per generare i figli.
Avendo però delle coppie la strategia cambia lievemente: sono scelte due coppie
come genitori dove, il primo Robby della prima coppia ed il primo Robby della
seconda generano il primo figlio, mentre il secondo Robby della prima coppia ed
il secondo della seconda coppia generano il secondo figlio. Per ognuno dei due
figli, il punto di taglio nel dna dei genitori varia.

\section{Mutazione Genetica}
La mutazione genetica è fatta nella maniera standard, mutando i geni dei due
robot generati in accordo con una percentuale fissata. Anche nel caso in cui
siano copiati degli individui dalla vecchia alla nuova generazione (senza dunque
l'uso del crossover), questi subiranno mutazioni.
