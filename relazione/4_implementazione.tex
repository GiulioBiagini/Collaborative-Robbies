\chapter{Implementazione}
In questo capitolo andremo a presentare gli strumenti utilizzati
nell'implementazione del codice. Spiegheremo anche come compilare ed eseguire lo
stesso, riportando come sono stati ottenuti i risultati delle simulazioni
presentati nel capitolo successivo.



\section{ANSI C ed MPI}
Il codice è stato scritto in \textbf{ANSI C} utilizzando la libreria
\textbf{MPI}, un'interfaccia per il \textit{Message Passing} open source. Grazie
ad essa l'eseguibile può essere inviato a più computer che ne eseguono
un'istanza. Ad ogni elaboratore è asseganto un id che può essere sfruttato in
modo da dare alla macchina un comportamento specifico.\newline
Nell'architettura da noi realizzata questi id sono usati per assegnare un
\textit{seed} ad ogni istanza del programma in esecuzione su ogni computer: si
parte con seed 10 e si sommano gli id. È così che il l'istanza del programma
eseguita dall'elaboratore con id 0 avrà seed 10, l'istanza del programma
eseguita dal computer con id 1 avrà seed 11 e così via\dots\newline
Le simulazioni da noi eseguite sono state lanciate con 11 elaboratori.



\section{Installazione}
Il programma è stato implementato e fatto eseguire su di un'architettura con
sistema operativo \textit{Debian stable 8.4}. Per compilarlo ed eseguirlo usando
MPI, è stato necessario installare il pacchetto \textit{mpich}.



\section{Compilazione}
La compilazione del sorgente avviene tramite il comando \textit{mpicc} a cui
abbiamo passato i flag \textit{--ansi}, \textit{-Wall}, \textit{-pedantic} e
\textit{-lm} per il linking della libreria matematica.



\section{Esecuzione}
L'esecuzione è lanciata tramite il comando \textit{mpirun} specificando con il
flag \textit{-np} il numero di processi da lanciare: 11 nel nostro caso. Questo
dato è molto importante al fine di ripetere le simulazioni da noi effettuate, in
quanto ogni processo esegue un'istanza del programma con un seed diverso dagli
altri, ovvero: \textit{SEED + id}, dove SEED è pari a 10.



\section{Architettura}
Il processo con id 0 agisce da master mentre gli altri 10 processi, con id da 1
a 10 appunto, agiscono da slave. Il master genera la popolazione iniziale e,
per ogni generazione, la divide in parti uguali ed invia ad ogni slave una parte
di coppie che devono essere valutate. Se ad esempio abbiamo una popolazione
formata da 200 coppie, ogni slave dovrà valutarne 20. Ogni slave valuta le
coppie a lui assegnate e risponde al master con i relativi valori di fitness.
Una volta ottenuti tutti i valori di fitness, il master ordina le coppie in base
a tali valori ed evolve la popolazione.\newline
Così facendo abbiamo notevolmente incrementato le prestazioni del nostro
programma. Esso infatti risulta essere distribuito ed esegue in parallelo 11
istanze, nonché 10 valutazioni di POPULATION\_SIZE/10 coppie. La velocità di
esecuzione è stata notevolmente aumentata.
