\documentclass[a4paper,12pt,openright,twoside]{book}

% libreria per scrivere in italiano
\usepackage[italian]{babel}

% libreria per accettare i caratteri utf-8
\usepackage[utf8]{inputenc}

% libreria per impostare il documento
\usepackage{fancyhdr}
\renewcommand{\chaptermark}[1]{\markboth{\thechapter.\ #1}{}}
\renewcommand{\sectionmark}[1]{\markright{\thesection \ #1}{}}
\rhead[\fancyplain{}{\bfseries\leftmark}]{\fancyplain{}{\bfseries\thepage}}
\cfoot{}

% per i collegamenti ipertestuali
\usepackage{hyperref}
\hypersetup{
	colorlinks=true,
	linkcolor=blue,
	citecolor=blue,
	filecolor=blue,
	urlcolor=blue
}

% per le immagini
\usepackage{graphicx}



% documento
\begin{document}

\begin{list}{}{
  \setlength{\topsep}{20pt}
  \setlength{\leftmargin}{-40pt}%
  \setlength{\rightmargin}{-80pt}%
  \setlength{\listparindent}{0pt}%
  \setlength{\itemindent}{0pt}%
  \setlength{\parsep}{0pt}%
 }%
\item[]
\thispagestyle{empty}

\begin{center}
{{\center \fontsize{17.4}{60}\selectfont \textsc{Alma Mater Studiorum $\cdot$ Università di Bologna}}}
\noindent\makebox[\linewidth]{\rule[0.1cm]{15.8cm}{0.1mm}}
\noindent\makebox[\linewidth]{\rule[0.5cm]{15.8cm}{0.6mm}}
{\small{\bf Dipartimento di Informatica - Scienza e Ingegneria (DISI)\\
Corso di Laurea Magistrale in Informatica}}

\end{center}
\vspace{40mm}
\begin{center}
{\LARGE{\bf Collaborative Robbies}}\\
%\vspace{3mm} {\large{\bf Remember when and thing won't be a problem anymore!}}\\
\vspace{3mm} {\large{\bf Relazione di FSC}}
\end{center}
\vspace{20mm}
\begin{center}
{\large{\bf 18 dicembre 2015}}
\end{center}
\vspace{50mm}
\par
\noindent
\begin{minipage}[t]{0.54\textwidth}\raggedright
{\large{\bf Giulio Biagini\\
0000705715\\
giulio.biagini@studio.unibo.it\vspace{\baselineskip}}}\\
\end{minipage}
\hfill
\begin{minipage}[t]{0.54\textwidth}\raggedleft
{\large{\bf Daniele Baschieri\\
0000688992\\
daniele.baschieri@studio.unibo.it}}
\end{minipage}

\end{list}

\clearpage
\newpage
\tableofcontents
\chapter{Introduzione}
In questo capitolo andremo a descrivere \textit{Robby: The Soda-Can-Collecting
Robot}~\cite{biblio:robby}, il progetto a cui questo lavoro si ispira,
riportando alcuni cenni teorici di Intelligenza Artificiale che useremo per
descrivere in maniera più formale la struttura dell'intero progetto.



\section{Robby: The Soda-Can-Collecting Robot}
Robby è un robot il cui compito consiste nel muoversi in uno spazio sporco e
cercare di ripulirlo. Lo spazio può essere visto in modo astratto come una
griglia suddivisa in celle. Robby possiede solamente una vista parziale dello
spazio nel quale si trova: è in grado di vedere cosa è presente nella cella a
nord, in quella a sinistra, nella casella in cui si trova, in quella a destra ed
in quella a sud rispetto alla propria posizione. Le celle possono essere
``pulite'' (vuote), ``sporche'' (nelle quali è presente una lattina da
raccogliere) oppure rappresentare un ostacolo (un muro).\newline
Robby attua un'azione in base alla vista che ha in quel momento dell'ambiente.
Se, ad esempio, vede una lattina nella cella a nord, un muro nella cella a
sinistra e niente nella cella da lui occupata, in quella a destra e nella cella
a sud, questo non si muoverà negli spazi vuoti, né tantomeno sbatterà contro il
muro muovendosi a sinistra, ma si muoverà di un passo verso l'alto, in direzione
della lattina, per poi raccoglierla. Le azioni che Robby può effettuare sono: la
mossa verso nord, verso sinistra, verso destra e verso sud, può decidere di
rimanere fermo, raccogliere una lattina oppure effettuare una mossa casuale
scelta fra le precedenti. Tutte le mosse hanno l'effetto immaginato a parte la
raccolta della lattina. Questa toglierà la lattina dalla mappa nella cella in
cui il robot si trova, se presente, altrimenti lascerà l'ambiente
invariato.\newline
Inizialmente Robby possiede un dna casuale, ovvero, ad ogni vista è associata
un'azione casuale. Robby è però in grado di apprendere come muoversi
correttamente nell'ambiente e come raccogliere lattine. La sua evoluzione è
infatti guidata da un Algoritmo Genetico.



\section{Entità Intelligenti}
Lo scopo dell'\textit{Intelligenza Artificiale} è quello di cercare di capire
come le \textit{entità intelligenti} funzionano ed, in particolare, quello di
cercare di costruire, creare, entità intelligenti.\newline
Negli anni sono state date varie definizioni di entità intelligenti: alcune
fanno paragoni con gli esseri umani, altre usano il principio della razionalità,
alcune definiscono queste entità in base a come pensano, altre a come agiscono.
Nel nostro particolare caso, andremo a definire un'entità come intelligente se
\textit{agisce in modo razionale}, ovvero se ``fa la cosa giusta''.\newline
Affinché \textbf{Robby}, l'entità intelligente che abbiamo il compito di
creare, possa essere definita intelligente, dovrà quindi ``fare la cosa
giusta''. Intuitivamente, siccome il compito del robot sarà quello di muoversi
in uno spazio cercando di pulirlo, ``fare la cosa giusta'' significherà, ad
esempio, non sbattere contro i muri durante il proprio movimento, raccogliere le
lattine durante il proprio passaggio e non tentare di raccogliere lattine dove
queste non sono presenti.



\section{Agenti Intelligenti}
Un \textit{agente intelligente} è un'entità intelligente in grado di percepire
l'ambiente tramite dei \textit{sensori} e compiere delle azioni tramite degli
\textit{attuatori}.\newline
Un agente ha delle \textit{percezioni} dell'ambiente, ovvero l'insieme di tutti
gli input percettivi che provengono dai propri sensori in un dato
istante.\newline
Il comportamento di un agente intelligente è descritto matematicamente da una
\textit{funzione agente}, ovvero l'insieme di tutte le sequenze percettive
e da tutte le relative azioni. L'implementazione di una funzione agente prende
il nome di \textit{programma agente}.\newline
Analizzando la definizione di agente intelligente, possiamo notare come
\textbf{Robby} sia un'entità che appartenga a questa categoria: esso infatti
è in grado di percepire l'ambiente tramite dei sensori che gli permettono di
vedere il contenuto delle celle che lo circondano e possiede degli attuatori che
gli consentono di muoversi e raccogliere lattine.\newline
L'insieme di tutte le viste (percezioni) che Robby può avere dell'ambiente e le
rispettive azioni sono descritte da una funzione agente. Il nostro compito sarà
quello di fornirne un'implementazione attraverso la scrittura di un programma
agente.



\section{Agenti Razionali}
Il nostro obiettivo, però, non è semplicemente quello di creare agenti
intelligenti, ovvero entità in grado di percepire l'ambiente e compiere azioni,
ma, piuttosto, creare \textit{agenti razionali}, ovvero entità sì intelligenti,
ma che ``facciano la cosa giusta''. Questo significa che il programma agente,
per ogni sequenza percettiva, produce un'azione che va a massimizzare una
determinata \textit{misura di prestazione}: se le azioni attuate dall'agente
portano l'ambiente ad attraversare una sequenza di stati che può essere definita
``desiderabile'', allora la misura di prestazione è massimizzata.\newline
Nel caso di \textbf{Robby}, la misura di prestazione da massimizzare sarà sia il
numero di lattine che debbono essere raccolte in un dato ambiente, sia il numero
di passi impiegato per raccoglierle.



\section{Agenti Reattivi Semplici}
Esistono varie tipologie di programmi agente in base alle tipologie di agenti
di cui questi hanno il compito di descrivere il comportamento. I più semplici di
tutti sono gli \textit{Agenti Reattivi Semplici}, ovvero agenti che basano le
proprie azioni solamente sulla percezione corrente. Questi agenti, dunque,
analizzano tutti gli input percettivi che provengono dai sensori in un dato
istante e computano quale azione compiere.\newline
\textbf{Robby} è un agente reattivo semplice in quanto la scelta dell'azione da
attuare è guidata solamente dalla vista che ha in quel momento dell'ambiente.



\section{Agenti in Grado di Apprendere}
Esistono particolari tipologie di agenti che possono essere programmati in modo
che siano \textit{in grado di apprendere}.\newline
Un agente, infatti, può calcolare la scelta delle proprie azioni basandosi su
conoscenze pregresse che ha dell'ambiente, che sono ad esempio state inserite a
priori da un programmatore: in questo caso si dice che l'agente manca di
autonomia. Al contrario un \textit{agente autonomo} è in grado di apprendere
per compensare la presenza di conoscenza parziale o erronea. Un agente in grado
di apprendere ha il vantaggio di poter operare in ambienti all'inizio
sconosciuti, diventando col tempo via via più competente.\newline
Questa tipologia di agenti possiedono, oltre ad un \textit{elemento esecutivo},
che gli permette di selezionare le azioni da compiere, anche un \textit{elemento
di apprendimento}, responsabile del miglioramento interno, il quale usa le
informazioni provenienti dall'\textit{elemento critico} riguardo le
prestazioni correnti dell'agente e determina se e come modificare l'elemento
esecutivo affinché in futuro si comporti meglio. Infine, le entità in grado di
apprendere possiedono un \textit{generatore di problemi}, il cui scopo è quello
di suggerire azioni che portino ad esperienze nuove e significative dalle quali
apprendere.\newline
Nel caso di \textbf{Robby}, quello che vogliamo è un agente autonomo in grado
di apprendere, ovvero un agente che non si basi su conoscenza pregressa da noi
inserita. Per fare questo ci appoggeremo sugli \textit{Algoritmi Genetici}.

\chapter{Obiettivo}
In questo capitolo sono descritti gli obiettivi di questo lavoro. In
particolare, quello che ci interessa non è tanto realizzare un sistema nel quale
un'entità si muove in un ambiete e cerca di ripulirlo dalle lattine, imparando
come fare grazie all'implementazione di un algoritmo genetico, ma studiare come,
inserendo due entità nell'ambiente, queste si influenzino a vicenda. Lo scopo
principale, dunque, è analizzare se può esservi \textbf{collaborazione}
spontanea o meno fra le due entità.

\section{Entità Invisibili - Vista a Croce}
Con la prima prova che andremo a fare, tenteremo di stabilire quali sono le
performance di due robot che si muovono sulla mappa senza vedersi, ovvero
ignorando le mosse dell'altro robot. Sarà dunque ammesso trovarsi nella stessa
cella e, nel caso in cui entrambi raccolgano una lattina posizionata in uno
stesso posto, il secondo a raccogliere non riceverà un punteggio
negativo.\newline
I due robot avranno vista a croce, ovvero potranno vedere la casella a nord,
quella a sinistra, quella nella quale si trovano, quella a destra e quella a
sud.

\section{Entità Invisibili - Vista Quadrata}
Come nel caso precedente, le due entità non saranno in grado di vedersi ma
avranno una percezione maggiore della mappa nella quale si trovano: potranno
infatti vedere uno spazio di 3x3 celle dove il robot occupa la posizione
centrale (seconda riga e seconda colonna).\newline
Questa prova ci permetterà di capire se potendo vedere una porzione di mappa
maggiore, i robot sono avvantaggiati nella pulizia della stessa, ovvero, se
disponendo di più ``risorse'' raggiungono un obiettivo che massimizza
maggiormente la misura di prestazione, sia essa la raccolta di un numero
maggiore di lattine, sia essa la raccolta dello stesso numero di lattine ma con
un numero minore di passi.

\section{Entità Visibili - Vista a Croce}
Con quest'ultima prova vedremo cosa succede dando la possibilità ai due robot
di vedersi. In questo caso, sarà punita la presenza di entrambi nella stessa
casella. I due robot dovranno dunque imparare a non tentare di occupare la
setssa cella.\newline
Vedremo così se i due robot impareranno a collaborare o se invece la presenza
di uno sotacolerò l'altro, raggiungendo valori minori della funzione di fitness.

\chapter{Risultati}


\end{document}
