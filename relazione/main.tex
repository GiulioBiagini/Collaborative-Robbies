\documentclass[a4paper,12pt,openright,twoside]{book}

% libreria per scrivere in italiano
\usepackage[italian]{babel}

% libreria per accettare i caratteri utf-8
\usepackage[utf8]{inputenc}

% libreria per impostare il documento
\usepackage{fancyhdr}
\renewcommand{\chaptermark}[1]{\markboth{\thechapter.\ #1}{}}
\renewcommand{\sectionmark}[1]{\markright{\thesection \ #1}{}}
\rhead[\fancyplain{}{\bfseries\leftmark}]{\fancyplain{}{\bfseries\thepage}}
\cfoot{}

% per i collegamenti ipertestuali
\usepackage{hyperref}
\hypersetup{
	colorlinks=true,
	linkcolor=blue,
	citecolor=blue,
	filecolor=blue,
	urlcolor=blue
}

% per le immagini
\usepackage{graphicx}



% documento
\begin{document}

\begin{list}{}{
  \setlength{\topsep}{20pt}
  \setlength{\leftmargin}{-40pt}%
  \setlength{\rightmargin}{-80pt}%
  \setlength{\listparindent}{0pt}%
  \setlength{\itemindent}{0pt}%
  \setlength{\parsep}{0pt}%
 }%
\item[]
\thispagestyle{empty}

\begin{center}
{{\center \fontsize{17.4}{60}\selectfont \textsc{Alma Mater Studiorum $\cdot$ Università di Bologna}}}
\noindent\makebox[\linewidth]{\rule[0.1cm]{15.8cm}{0.1mm}}
\noindent\makebox[\linewidth]{\rule[0.5cm]{15.8cm}{0.6mm}}
{\small{\bf Dipartimento di Informatica - Scienza e Ingegneria (DISI)\\
Corso di Laurea Magistrale in Informatica}}

\end{center}
\vspace{40mm}
\begin{center}
{\LARGE{\bf Collaborative Robbies}}\\
%\vspace{3mm} {\large{\bf Remember when and thing won't be a problem anymore!}}\\
\vspace{3mm} {\large{\bf Relazione di FSC}}
\end{center}
\vspace{20mm}
\begin{center}
{\large{\bf 18 dicembre 2015}}
\end{center}
\vspace{50mm}
\par
\noindent
\begin{minipage}[t]{0.54\textwidth}\raggedright
{\large{\bf Giulio Biagini\\
0000705715\\
giulio.biagini@studio.unibo.it\vspace{\baselineskip}}}\\
\end{minipage}
\hfill
\begin{minipage}[t]{0.54\textwidth}\raggedleft
{\large{\bf Daniele Baschieri\\
0000688992\\
daniele.baschieri@studio.unibo.it}}
\end{minipage}

\end{list}

\clearpage
\newpage
\tableofcontents
\chapter{Introduzione}
In questo capitolo andremo a richiamare alcuni cenni teorici di Intelligenza
Artificiale che useremo per descrivere il comportamento dell'entità che abbiamo
il compito di progettare, affinché possa compiere nel migliore dei modi il
proprio dovere.

\section{Entità Intelligenti}
Lo scopo dell'\textit{Intelligenza Artificiale} è quello di cercare di capire
come le \textit{entità intelligenti} funzionano ed, in particolare, quello di
cercare di costruire, creare, entità intelligenti.\newline
Negli anni sono state date varie definizioni di entità ``intelligenti'': alcune
fanno paragoni con gli esseri umani, altre usano il principio della razionalità,
alcune definiscono queste entità in base a come pensano, altre a come agiscono.
Nel nostro particolare caso andremo a definire un'entità come intelligente se
\textit{agisce in modo razionale}, ovvero se ``fa la cosa giusta''.\newline
Affinché \textbf{Robby}, l'entità intelligente che abbiamo il compito di
progettare, possa essere definita come tale, dovrà dunque ``fare la cosa
giusta''. Siccome questa dovrà muoversi in uno spazio cercando di pulirlo,
questo significherà, ad esempio, non sbattere contro i muri, raccogliere le
lattine durante il proprio passaggio o non tentare di raccogliere lattine dove
queste non ci sono.

\section{Agenti Intelligenti}
Un \textit{agente intelligente} è un'entità intelligente in grado di percepire
l'ambiente tramite dei \textit{sensori} e compiere delle azioni tramite degli
\textit{attuatori}.\newline
Un agente ha delle \textit{percezioni} dell'ambiente, ovvero l'insieme di tutti
gli input percettivi che provengono dai propri sensori in un dato
istante.\newline
Il comportamento di un agente intelligente è descritto matematicamente da una
\textit{funzione agente}, ovvero l'insieme di tutte le sequenze percettive
e da tutte le relative azioni. L'implementazione di una funzione agente prende
il nome di \textit{programma agente}.\newline
Il nostro primo obiettivo, dunque, sarà quello di dotare \textbf{Robby} di
sensori che gli permettano di percepire l'ambiente ed attuatori che faranno sì
che esso potrà muoversi e compiere determinate azioni.\newline
I sensori di cui Robby sarà dotato gli permetteranno di percepire lo spazio
attorno, ovvero, la presenza muri, lattine o se esso è vuoto. Gli attuatori gli
permetteranno di muoversi verso nord, verso sud, a destra, a sinistra e di
rimanere fermo nella posizione in cui si trova. Robby avrà poi speciali
attuatori che faranno sì che possa raccogliere lattine.\newline
Il programma agente che specificherà al robot quale azione compiere in base alla
vista che avrà in quel momento (percezione dell'ambiente) caratterizzerà il
comportamento del robot stesso.

\section{Agenti Razionali}
Lo scopo dell'Intelligenza Artificiale, però, non è semplicemente quello di
andare a creare agenti intelligenti, ovvero entità in grado di percepire
l'ambiente ed attuarvi azioni, ma, piuttosto, creare \textit{agenti
razionali}, ovvero entità intelligenti che ``facciano la cosa giusta''.\newline
Questo significa che il programma agente per ogni sequenza percettiva produce
un'azione che massimizza una determinata \textit{misura di prestazione}: se le
azioni attuate portano l'ambiente ad attraversare una sequenza di stati che
potranno essere definiti ``desiderabili'', allora il programma agente massimizza
tale misura di prestazione.\newline
Nel caso di \textbf{Robby}, la misura di prestazione da massimizzare sarà il
numero di lattine che saranno mediamente raccolte in un dato ambiente.

\section{Agenti Reattivi Semplici}
Esistono varie tipologie di programmi agente in base alle tipologie di agenti
di cui questi hanno il compito di descrivere il comportamento. I più semplici di
tutti sono gli \textit{Agenti Reattivi Semplici}, ovvero agenti che basano le
proprie azioni solamente sulla percezione corrente: in base a quanto letto dagli
input percettivi in un dato istante, è calcolata l'azione da compiere.\newline
\textbf{Robby}, che sarà in grado di ``vedere'' una porzione dello spazio che lo
circonda, attuerà delle azioni che saranno guidate solamente da quanto percepito
in quel dato momento. Esso sarà dunque un agente reattivo semplice.

\section{Ambiente}
L'\textit{ambiente} nel quale gli agenti razionali si muovono possono essere di
vario tipo in base alle caratteristiche che possiedono.\newline
Lo spazio nel quale \textbf{Robby} dovrà muoversi sarà discretizzato in celle,
ogni cella potrà essere occupata da un ostacolo (muro), da una lattina o potrà
essere vuota. Le azioni di movimento permetteranno al robot di muoversi da una
cella all'altra mentre, le lattine potranno essere raccolte solamente se il
robot si troverà su una casella da esse occupata.\newline
L'ambiente sarà \textit{parzialmente osservabile}, ovvero solo il contenuto di
alcune celle potrà essere rilevato e \textit{stocastico}, in quanto è presente
l'azione random che permetterà ai robot di poter attuare una delle azioni
precedentemente elencate scelta casualmente.

\section{Agenti in Grado di Apprendere}
Esistono particolari tipologie di agenti che possono essere programmate in modo
che siano \textit{in grado di apprendere}. Un agente, infatti, può basare la
scelta delle proprie azioni su conoscenze che ha dell'ambiente che sono state
inserite da un programmatore: in questo caso si dice che l'agente manca di
autonomia. Al contrario, un \textit{agente autonomo} è in grado di apprendere
per compensare la presenza di conoscenza parziale o erronea.\newline
Un agente in grado di apprendere ha il vantaggio di poter operare in ambienti
all'inizio sconosciuti, diventando col tempo via via più competente.\newline
Un agente in grado di apprendere possiede oltre ad un \textit{elemento
esecutivo}, che gli permette di selezionare le azioni da compiere, anche di un
\textit{elemento di apprendimento}, responsabile del miglioramento interno, il
quale usa le informazioni provenienti dall'\textit{elemento critico} riguardo le
prestazioni correnti dell'agente e determina se e come modificare l'elemento
esecutivo affinché in futuro si comporti meglio ed un \textit{generatore di
problemi} il cui scopo è quello di suggerire azioni che portino ad esperienze
nuove e significative dalle quali apprendere.\newline
Nel caso di \textbf{Robby}, quello che vogliamo non è solo un agente autonomo,
ma anche un agente in grado di apprendere. Per fare questo ci appoggeremo ad
\textit{Algoritmi Genetici}.

\section{Algoritmi Genetici}
Gli \textit{Algoritmi Genetici} sono algoritmi che sono spesso usati in quelle
situazioni nelle quali risulta difficile andare a progettare una determinata
strategia che ci permetta di arrivare ad una soluzione.\newline
Questi algoritmi sono difatti usati per far sì che siano loro stessi ad evolvere
il sistema in modo che esso arrivi autonomamente ad una soluzione.\newline
Uno dei campi nei quali sono maggiormente impiegati, infatti, sono i
\textit{Sistemi Complessi}.\newline
Per far sì che \textbf{Robby} possa muoversi e pulire l'ambiente raccogliendo il
maggior numero di lattine con il minor numero di mosse, non conoscendo a priori
la posizione del robot stesso nella mappa, né delle lattine, non rimane che
progettare un algoritmo genetico che permetta ai robot di apprendere una buona
strategia.\newline
Dapprima sono generati \textit{n} individui con un \textit{dna} casuale, ovvero,
ad ogni possibile vista è associata una mossa casuale. Dopodiché si valuta
ciascun robot per un numero \textit{p} di passi su di una mappa nella quale le
posizioni delle lattine sono generate casualmente, così come la posizione del
robot. Si ripete la valutazione su \textit{m} mappe per ogni robot assegnando,
alla fine, un \textit{valore di fitness} medio a ciascuna entità. Questo valore
è stabilito da una \textit{funzione di fitness} che valuta ogni mossa del robot:
se questo muove contro un ostacolo viene punito con un valore negativo, se tenta
di raccogliere una lattina dove essa non è presente viene altresì puntio,
mentre, nel caso in cui una lattina sia raccolta viene premiato. Alla fine della
fase di valutazione, i robot sono ordinati in base al valore di fitness, ed ha
inizio la generazione della nuova popolazione: sono scelti due individui in base
al valore di fitness o in base alla posizione nel ranking globale e viene
attuato il \textit{crossover}. Questa metodologia consiste nella scelta di un
punto casuale del dna dei due genitori e, unendo rispettivamente la prima metà
del dna del primo genitore con la seconda metà del dna del secondo genitore
viene generato il primo figlio ed unendo la seconda metà del dna del primo
genitore con la prima metà del dna del secondo genitore viene originato il
secondo figlio. Infine sono presi tutti i \textit{geni} (vista/azione) di ogni
figlio e con una probabilità \textit{x} sono \textit{mutati}, cioè, alla vista
è modificata in modo casuale l'azione corrispondente. Questo procedimento di
generazione di una nuova generazione a partire dalla vecchia che viene poi a sua
volta valutata, viene iterato \textit{g} volte. È così che, dopo svariate
generazioni, i robot apprendono strategie di pulizia che permettono loro di
arrivare molto vicino all'obiettivo.

\chapter{Obiettivo}
In questo capitolo sono descritti gli obiettivi di questo lavoro. In
particolare, quello che ci interessa non è tanto realizzare un sistema nel quale
un'entità si muove in un ambiete e cerca di ripulirlo dalle lattine, imparando
come fare grazie all'implementazione di un algoritmo genetico, ma studiare come,
inserendo due entità nell'ambiente, queste si influenzino a vicenda. Lo scopo
principale, dunque, è analizzare se può esservi \textbf{collaborazione}
spontanea o meno fra le due entità.

\section{Entità Invisibili - Vista a Croce}
Con la prima prova che andremo a fare, tenteremo di stabilire quali sono le
performance di due robot che si muovono sulla mappa senza vedersi, ovvero
ignorando le mosse dell'altro robot. Sarà dunque ammesso trovarsi nella stessa
cella e, nel caso in cui entrambi raccolgano una lattina posizionata in uno
stesso posto, il secondo a raccogliere non riceverà un punteggio
negativo.\newline
I due robot avranno vista a croce, ovvero potranno vedere la casella a nord,
quella a sinistra, quella nella quale si trovano, quella a destra e quella a
sud.

\section{Entità Invisibili - Vista Quadrata}
Come nel caso precedente, le due entità non saranno in grado di vedersi ma
avranno una percezione maggiore della mappa nella quale si trovano: potranno
infatti vedere uno spazio di 3x3 celle dove il robot occupa la posizione
centrale (seconda riga e seconda colonna).\newline
Questa prova ci permetterà di capire se potendo vedere una porzione di mappa
maggiore, i robot sono avvantaggiati nella pulizia della stessa, ovvero, se
disponendo di più ``risorse'' raggiungono un obiettivo che massimizza
maggiormente la misura di prestazione, sia essa la raccolta di un numero
maggiore di lattine, sia essa la raccolta dello stesso numero di lattine ma con
un numero minore di passi.

\section{Entità Visibili - Vista a Croce}
Con quest'ultima prova vedremo cosa succede dando la possibilità ai due robot
di vedersi. In questo caso, sarà punita la presenza di entrambi nella stessa
casella. I due robot dovranno dunque imparare a non tentare di occupare la
setssa cella.\newline
Vedremo così se i due robot impareranno a collaborare o se invece la presenza
di uno sotacolerò l'altro, raggiungendo valori minori della funzione di fitness.

\include{3_risultati}

\end{document}
