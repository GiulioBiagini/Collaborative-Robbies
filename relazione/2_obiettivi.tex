\chapter{Obiettivi}
In questo capitolo sono descritti gli obiettivi di questo lavoro. In
particolare, quello che ci interessa non è tanto re-implementare il lavoro
proposto da Melanie Mitchell~\cite{biblio:robby}, quanto piuttosto realizzare un
sistema che ci permetta di studiare come, inserendo più entità nell'ambiente,
queste si influenzino a vicenda.\newline
Lo scopo principale, dunque, è analizzare se può esservi \textbf{collaborazione
spontanea} o meno fra le entità che si trovano a nella stessa mappa.



\section{Entità Invisibili - Vista a Croce}
Come prima cosa andremo a cercare di stabilire quali sono le performance di due
agenti che si muovono sulla mappa senza vedersi, ovvero ignorando la posizione
dell'altro robot. Sarà dunque ammesso trovarsi nella stessa cella e raccogliere
la stessa lattina.\newline
I due robot avranno vista a croce, ovvero come quella descritta nel capitolo
introduttivo: potranno vedere la casella a nord, quella a sinistra, quella nella
quale si trovano, quella a destra e quella a sud.\newline
I risultati ottenuti saranno poi usati come ``caso base'', per fare un confronto
con le simulazioni lanciate successivamente.



\section{Entità Invisibili - Vista Quadrata}
In questo secondo test, così come nel caso precedente, le due entità non saranno
in grado di vedersi ma avranno una percezione maggiore della mappa nella quale
si troveranno. Potranno infatti vedere uno spazio di 3x3 celle dove il robot
occupa la posizione centrale (seconda riga e seconda colonna).\newline
Questa prova ci permetterà di capire se potendo vedere una porzione di mappa
maggiore, i robot saranno avvantaggiati nella pulizia della stessa, ovvero, se
disponendo di più risorse (sensori più potenti) saranno in grado di massimizzare
maggiormente la misura di prestazione: sia essa la raccolta di un numero
maggiore di lattine, sia essa la raccolta dello stesso numero di lattine ma con
un numero minore di passi.



\section{Entità Visibili - Vista a Croce}
Con quest'ultima simulazione vedremo cosa succede dando la possibilità ai due
robot di vedersi. In questo caso non sarà tollerata la compresenza nella
medesima cella.\newline
Vedremo così se i due robot impareranno a collaborare o se invece la presenza
di uno ostacolerà l'altro.
