\chapter{Obiettivi}
In questo capitolo sono descritti gli obiettivi di questo lavoro. In
particolare, quello che ci interessa non è tanto realizzare un sistema nel quale
un'entità si muove in un ambiete e cerca di ripulirlo dalle lattine, imparando
come fare grazie all'implementazione di un algoritmo genetico, ma studiare come,
inserendo due entità nell'ambiente, queste si influenzino a vicenda. Lo scopo
principale, dunque, è analizzare se può esservi \textbf{collaborazione}
spontanea o meno fra le due entità.

\section{Entità Invisibili - Vista a Croce}
Con la prima prova che andremo a fare, tenteremo di stabilire quali sono le
performance di due robot che si muovono sulla mappa senza vedersi, ovvero
ignorando le mosse dell'altro robot. Sarà dunque ammesso trovarsi nella stessa
cella e, nel caso in cui entrambi raccolgano una lattina posizionata in uno
stesso posto, il secondo a raccogliere non riceverà un punteggio
negativo.\newline
I due robot avranno vista a croce, ovvero potranno vedere la casella a nord,
quella a sinistra, quella nella quale si trovano, quella a destra e quella a
sud.

\section{Entità Invisibili - Vista Quadrata}
Come nel caso precedente, le due entità non saranno in grado di vedersi ma
avranno una percezione maggiore della mappa nella quale si trovano: potranno
infatti vedere uno spazio di 3x3 celle dove il robot occupa la posizione
centrale (seconda riga e seconda colonna).\newline
Questa prova ci permetterà di capire se potendo vedere una porzione di mappa
maggiore, i robot sono avvantaggiati nella pulizia della stessa, ovvero, se
disponendo di più ``risorse'' raggiungono un obiettivo che massimizza
maggiormente la misura di prestazione, sia essa la raccolta di un numero
maggiore di lattine, sia essa la raccolta dello stesso numero di lattine ma con
un numero minore di passi.

\section{Entità Visibili - Vista a Croce}
Con quest'ultima prova vedremo cosa succede dando la possibilità ai due robot
di vedersi. In questo caso, sarà punita la presenza di entrambi nella stessa
casella. I due robot dovranno dunque imparare a non tentare di occupare la
setssa cella.\newline
Vedremo così se i due robot impareranno a collaborare o se invece la presenza
di uno sotacolerò l'altro, raggiungendo valori minori della funzione di fitness.
